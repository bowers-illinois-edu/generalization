% Options for packages loaded elsewhere
\PassOptionsToPackage{unicode}{hyperref}
\PassOptionsToPackage{hyphens}{url}
\PassOptionsToPackage{dvipsnames,svgnames,x11names}{xcolor}
%
\documentclass[
]{article}
\usepackage{amsmath,amssymb}
\usepackage{lmodern}
\usepackage{iftex}
\ifPDFTeX
  \usepackage[T1]{fontenc}
  \usepackage[utf8]{inputenc}
  \usepackage{textcomp} % provide euro and other symbols
\else % if luatex or xetex
  \usepackage{unicode-math}
  \defaultfontfeatures{Scale=MatchLowercase}
  \defaultfontfeatures[\rmfamily]{Ligatures=TeX,Scale=1}
\fi
% Use upquote if available, for straight quotes in verbatim environments
\IfFileExists{upquote.sty}{\usepackage{upquote}}{}
\IfFileExists{microtype.sty}{% use microtype if available
  \usepackage[]{microtype}
  \UseMicrotypeSet[protrusion]{basicmath} % disable protrusion for tt fonts
}{}
\makeatletter
\@ifundefined{KOMAClassName}{% if non-KOMA class
  \IfFileExists{parskip.sty}{%
    \usepackage{parskip}
  }{% else
    \setlength{\parindent}{0pt}
    \setlength{\parskip}{6pt plus 2pt minus 1pt}}
}{% if KOMA class
  \KOMAoptions{parskip=half}}
\makeatother
\usepackage{xcolor}
\IfFileExists{xurl.sty}{\usepackage{xurl}}{} % add URL line breaks if available
\IfFileExists{bookmark.sty}{\usepackage{bookmark}}{\usepackage{hyperref}}
\hypersetup{
  pdfauthor={Jake Bowers; James Kuklinski},
  colorlinks=true,
  linkcolor={Maroon},
  filecolor={Maroon},
  citecolor={Blue},
  urlcolor={Blue},
  pdfcreator={LaTeX via pandoc}}
\urlstyle{same} % disable monospaced font for URLs
\usepackage[margin=1in]{geometry}
\usepackage{color}
\usepackage{fancyvrb}
\newcommand{\VerbBar}{|}
\newcommand{\VERB}{\Verb[commandchars=\\\{\}]}
\DefineVerbatimEnvironment{Highlighting}{Verbatim}{commandchars=\\\{\}}
% Add ',fontsize=\small' for more characters per line
\newenvironment{Shaded}{}{}
\newcommand{\AlertTok}[1]{\textcolor[rgb]{1.00,0.00,0.00}{\textbf{#1}}}
\newcommand{\AnnotationTok}[1]{\textcolor[rgb]{0.38,0.63,0.69}{\textbf{\textit{#1}}}}
\newcommand{\AttributeTok}[1]{\textcolor[rgb]{0.49,0.56,0.16}{#1}}
\newcommand{\BaseNTok}[1]{\textcolor[rgb]{0.25,0.63,0.44}{#1}}
\newcommand{\BuiltInTok}[1]{#1}
\newcommand{\CharTok}[1]{\textcolor[rgb]{0.25,0.44,0.63}{#1}}
\newcommand{\CommentTok}[1]{\textcolor[rgb]{0.38,0.63,0.69}{\textit{#1}}}
\newcommand{\CommentVarTok}[1]{\textcolor[rgb]{0.38,0.63,0.69}{\textbf{\textit{#1}}}}
\newcommand{\ConstantTok}[1]{\textcolor[rgb]{0.53,0.00,0.00}{#1}}
\newcommand{\ControlFlowTok}[1]{\textcolor[rgb]{0.00,0.44,0.13}{\textbf{#1}}}
\newcommand{\DataTypeTok}[1]{\textcolor[rgb]{0.56,0.13,0.00}{#1}}
\newcommand{\DecValTok}[1]{\textcolor[rgb]{0.25,0.63,0.44}{#1}}
\newcommand{\DocumentationTok}[1]{\textcolor[rgb]{0.73,0.13,0.13}{\textit{#1}}}
\newcommand{\ErrorTok}[1]{\textcolor[rgb]{1.00,0.00,0.00}{\textbf{#1}}}
\newcommand{\ExtensionTok}[1]{#1}
\newcommand{\FloatTok}[1]{\textcolor[rgb]{0.25,0.63,0.44}{#1}}
\newcommand{\FunctionTok}[1]{\textcolor[rgb]{0.02,0.16,0.49}{#1}}
\newcommand{\ImportTok}[1]{#1}
\newcommand{\InformationTok}[1]{\textcolor[rgb]{0.38,0.63,0.69}{\textbf{\textit{#1}}}}
\newcommand{\KeywordTok}[1]{\textcolor[rgb]{0.00,0.44,0.13}{\textbf{#1}}}
\newcommand{\NormalTok}[1]{#1}
\newcommand{\OperatorTok}[1]{\textcolor[rgb]{0.40,0.40,0.40}{#1}}
\newcommand{\OtherTok}[1]{\textcolor[rgb]{0.00,0.44,0.13}{#1}}
\newcommand{\PreprocessorTok}[1]{\textcolor[rgb]{0.74,0.48,0.00}{#1}}
\newcommand{\RegionMarkerTok}[1]{#1}
\newcommand{\SpecialCharTok}[1]{\textcolor[rgb]{0.25,0.44,0.63}{#1}}
\newcommand{\SpecialStringTok}[1]{\textcolor[rgb]{0.73,0.40,0.53}{#1}}
\newcommand{\StringTok}[1]{\textcolor[rgb]{0.25,0.44,0.63}{#1}}
\newcommand{\VariableTok}[1]{\textcolor[rgb]{0.10,0.09,0.49}{#1}}
\newcommand{\VerbatimStringTok}[1]{\textcolor[rgb]{0.25,0.44,0.63}{#1}}
\newcommand{\WarningTok}[1]{\textcolor[rgb]{0.38,0.63,0.69}{\textbf{\textit{#1}}}}
\setlength{\emergencystretch}{3em} % prevent overfull lines
\providecommand{\tightlist}{%
  \setlength{\itemsep}{0pt}\setlength{\parskip}{0pt}}
\setcounter{secnumdepth}{2}
\usepackage{fancyhdr,quiver}
\fancypagestyle{myfancy}{%
  \fancyfoot[C]{}
  \fancyfoot[R]{Version of --- \today --- \thepage}}
\pagestyle{myfancy}
\ifLuaTeX
  \usepackage{selnolig}  % disable illegal ligatures
\fi
\usepackage[]{biblatex}
\addbibresource{../../Research-Group-Bibliography/big.bib}

\title{``Explanation, Generalization, Decision: The Role of Cumulation
in Policy Design and Decision and the Role of PAPs and Research
Integrity Practices there in. Also Theory and Explanation.''}
\author{Jake Bowers\footnote{\href{mailto:jwbowers@illinois.edu}{\nolinkurl{jwbowers@illinois.edu}}} \and James
Kuklinski}
\date{Version of 2021-Sep-07}

\begin{document}
\maketitle

Key points: - Any one or several studies helps policy makers make
decisions to the extent that they fill in the theory of change held
implicitly or explicitly by the decision maker. Imagien a theory of
change as a DAG. If a study informs me something new about a path over
which I have control / could make a change / am considering a change,
then I might use that study.

\begin{itemize}
\item
  For an academic: ``Does this study generalize?'' means ``I am having
  trouble fitting the results of this study into my DAG.'' and ``I would
  like to think that I could use this finding to reason about what I
  might expect in another context should I be asked to advise a decision
  maker.'' Or ``I would like to use this finding to add confidence in
  parts of the DAG that the scientific community is creating.''
\item
  For a decision maker (or group of them, like a city council): Care
  less about generalizing per se, and more about what a given study or
  set of them have to say about possibilites in a given place.
\end{itemize}

Imagine that that only one Providence Talks replication study was able
to be completed and it was the one in Birminham, Alabama. Imagine it
showed that families receiving the coaching plus feedback about the
device ended up speaking more with their child at the end, and had more
conversations with their child. Imagine that it also showed that those
children had higher scores in kindergarten readiness than children in
the control group. What might this mean for you if you were considering
spending effort on policies focused on low income children? (Knowing
that you might be consumed with policies to fix roads, etc.. and that
low income children kindergarten readiness while important of course
might not always be the primary policy priority.)

What about if all 5 cities had been able to complete the project and
that more or less the same results were seen across them all? Would that
change anything?

What kind of study not in your own city would be most useful?

Issues: We may not all share a DAG. Even if we do share a DAG certain
variables are fixed (too difficult to change) and others are not --- and
this varies across places. Arrows from fixed variables don't help me
much as a policy maker.

So: - COuld be worth asking for DAGs from policy decision makers before
you do one or even coordinated studies. Studies that randomize something
that no decision maker would change are not helpful to that person, for
example. - Also could be that evidence (say, RCTs) can inform policy
even if those RCTs do not evaluate a whole policy in an ITT form. An RCT
that provides focused information about one arrow might be more useful
than an RCT that provides an ITT style result, for example. - Role of
context: similar to the fixed variables. Often these are implicit.
Evaluate effect of the kind of policy intervention that is meaningful
right now --- postcards, say. But this assumes mailboxes and a mailing
service and literacy. Importance of trying to figure this out ---
pre-mortems, dark logic, participatory DAG creation, mobilizing
creativity. - Another implication: scientific community kind of provides
the DAGs. What if we have a DAG that says ``Stricking Match
--\textgreater{} Friction --\textgreater{} Flame'' makes sense. Are we
wrong if someone tries this on the Moon and discovered that it does
work? In fact we only have a partial story. And the failure to get flame
on the moon from striking a match is useful --- it reveals that we need
consider what else we should add to the DAG (and why, what is it about
Friction, etc.. that causes flame?)

For the policy maker the question is not whether one can predict the
effect of doing the intervention as done in City B in her own City A.
This is pretty rare. Mostly the way that the intevention would occur in
City A will differ. Learning about a similar intervention and similar
context in City B should help. And perhaps a City C that is like CIty B
might also help. And a City D that is ?different from either B or C?
might also help? But this makes this task a bit different from the task
focused on by, say, Pearl and Bareimboim (2014, 2016). That task might
be useful and may well add a lot --- but I suspect it adds more because
of the DAG than because of the prediction.

(from a persentation by Cinelli and Bareimboim)

\begin{quote}
``\,`External validity' asks the question of generalizability: To what
populations, settings, treatment variables, and measurement variables
can this effect be generalized?'' • Shadish, Cook and Campbell (2002)
\end{quote}

\begin{quote}
``Extrapolation across studies requires `some understanding of the
reasons for the differences.'\,'' • Cox (1958) ``An experiment is said
to have ``external validity'' if the distribution of outcomes realized
by a treatment group is the same as the distribution of outcome that
would be realized in an actual program.'' Manski (2007)
\end{quote}

What does a policy expert making decisions for the good of the people in
city A tomorrow learn from a study done in city B yesterday or last year
or ten years ago? What should such a person learn from other studies in
other cities in other times? What research practices might help this
person learn better / use the public / published information more
effectively or less effectively?

We engage with these questions in an effort to help (1) help policy
oriented research teams adapt research integrity practices from the
social sciences to their context and (2) help those who are considering
coordinated experiments or studies plan them so as to most effectively
enhance evidence-based policy decision making.

Overall points: - Policy decisions should help a fixed set of people in
a fixed place over a more or less fixed time (perhaps time is most
flexible). If I am in charge of helping families with young children
enhance the language development of those toddlers in a given place,
that is my job, my focus. I am not hoping to improve the lives of people
in other places or other times except very secondarily as people learn
about what I do. And, in fact, teaching others about what I do is
secondary or tertiary. If by chance people outside my place learn, then
fine. But it is no where near the most important thing. - This means
that learning that the estimated causal effect of an intervention is 100
words plus or minus 20 words (for example) in one other city can only be
used in terms of \textbf{sign} not magnitude. What matters in fact is
that it would be very surprising to see 100 if the null were true or
that we ``can detect an effect''. (More decision makers should pay more
attention to magnitude -- after all, a detectable effect of 10 words may
not be worth the expense of the program. But these conversations ---
``how large of an effect would be necessary to justify the expense of
the program'' have turned out to be very difficult in many areas --- it
is hard to monetize, say, academic inequality or equality. We set this
aside for now but may be an important task to tackle in the future ---
how to help people reason about whether a study would be worthwhile
given different costs of programs and anticipated benefits.) - This also
means that ``statistical significance'' seems to matter --- people do
not have well development judgements about continuous p-values. Or, say,
``posterior probability that the effect is greater than 0''. - So, savvy
decision makers are not looking for forecasts in most cases --- they
understand that (1) the ways in which their city will provide the
intervention differs from other cities (in statistical terms, the nature
of the treatment will differ), the measurement of the outcome will
differ, the ways in which the context of the city facilitates or hinders
the causal mechanism linking intervention to outcome differs in their
city from other cities. (This is not so different from family decision
makers by the way. Of course we know that, on average 5 servings of
fruit or vegetables per day is associated with better health outcomes
for kids. Yet, we don't imagine that this is a forecast for a given
family. It is better than nothing but each family knows it is not the
average family and so must make its own decisions in regards fruit and
vegetable servings.)

So, one other study provides information to the extent that it comes
from similar places (i.e.~similar covariate distributions) and perhaps a
sense that the intervention to covariate interactions are similar, and
intervention provision is similar, and outcome measurement and scale is
similar.

What about effects found in 5 cities that are not particularly similar
to ones own city? If those 5 were all the same basically, then it would
be just like one big study of a dissimilar city: bigger study is better
for sure. If those 5 differed from the focal city but also from each
other, is this better information? (I think so.) Would ``statistically
significant'' in all of those cities warrant action in the focal city?
Maybe. Unless there are some key moderators that differ: maybe the
causal mechanism only functions well under certain circumstances ---
say, if there is no civil war in the city; or the coaches are native
speakers of the home language of the family (for example, a growing
number of families in the USA are categorized as Spanish speakers as
coming from Guatemala but are in fact speakers of Mayan dialects in the
home.)

What about credibility? Why should a decision maker believe that the
results are what they are? Policy decisions are often politicized or at
least occur in the context of alternatives: should we spend the money on
coaching and LENA devices or just give cash to families or build more
libraries near these people? If the results appear to have come from a
research team with a particular political point of view, what does that
mean?

Should a decision maker in City A pay to support a study in Cities B,C,
and D? even if those cities differ from City in key ways?

What if the decision maker instead pays for a study in City A itself.
Does this result provide clear avenues for action? It might. What would
studies in other cities add here? (I'd say they would add some
information about how the intervention might change if the context
changes --- and since the context will change, a savvy decision maker
would want to implement a policy knowning this, perhaps planning in
advance for such changes.) For example, flood prevention in cities that
have not yet had serious floods --- engineers in those cities look to
other cities that have had recent floods to learn. Houston is not
Chicago, but the flooding of Houston from the Hurricane Blah can teach
enginners in Chicago about their storm sewers.

Notice also: decisions tend to emerge from committees with lots of
influence from stakeholders as well as a few influential voices. So,
there tends not to be a single decision maker.

I think that first a city council (say) is adding up advantages and
disadvantages of any given new idea --- say, spending money to deploy
coaches to provide feedback to parents about words spoken in their home
as recorded by some privacy protecting technology. Some of the
advantages and disadvantages are about how such an intervention might be
received/interpreted by those who do not directly benefit but who are
constituents: why not spend the money on splash parks, we know that
exercise is good and more kids can access this, not just the few poorest
children? Why not just give half of the money allocated to coaches and
devices to the families directly and the other half to splash parks ---
a compromise, families can even be told that they can spend their money
on LENA devices and coaching online.

What can studies done elsewhere add to this discussion? Imagine the
debate is between three mechanisms to improve the lives of low income
toddlers: exercise and the outdoors (could be over 1 year if we want to
staff the splash parks, or spend to advertise them, etc..), feedback
about home language over about 1 year period, and moderately increased
income over a 1 year period,

What if a decision maker group in City A could draw a DAG. And a group
of researchers could claim to have learned about a causal effect

\hypertarget{introduction-motivation-overview}{%
\section{Introduction, Motivation,
Overview}\label{introduction-motivation-overview}}

\textbf{The public policy problem:} Young children who grow up in low
socioeconomic status households score lower on reading related
assessments, on average, than young children growing up in higher SES
households. This difference can make it harder, in principle, for a
society to provide an equal opportunity to both kinds of children: they
may attend the same school, but one group arrives behind the other group
in regards reading and language. In this case, equal schools alone
cannot provide equal opportunity.

\textbf{Theories of language development:} Motivated by such persistent
correlations and concerns about equality and justice, researchers
suggest several mechanisms by which low SES children arrive in
kindergarten less able to read than higher SES children by looking at
pre-kindergarten differences between high and low SES children. Such
children differ, for example, in their nutrition, the extent to which
they may be exposed to neighborhood or domestic violence, and also the
number of words spoken to them. Scholars have proposed different
socio-neurological mechanisms linking these and other differences to the
end point of lower reading scores among the poor than among the better
off. For example, hearing words spoken to you may increase your hearing
vocabulary which in turn can make it easier to map meaning onto words on
the page.

\textbf{A public policy proposal:} A team of researchers has been
pursuing the idea that by improving the home language environment ---
words spoken to a child, books read to a child, two-way conversations
with a child --- the ``word gap'' that maps onto the SES gap can be
diminished or erased --- and that thus, the early elementary school
academic achievement gap can be diminished or erased.

\textbf{A second order theory of behavior change:} To change children's
vocabularies and pre-kindergarten language development these researchers
realized that they have to change the behavior of adults, and especially
parents and teachers of toddlers in day-care facilities. So, they had a
second order problem --- how to change how parents and other adults use
language around a toddler. Behavior change is hard and much studied. The
mechanism that these folks decided to focus on involved feedback ---
just as weight-loss behavior can be reinforced by using a scale, they
proposed to measure words spoken to children and two-way conversations
with children in the home (and childcare classrooms) and then to reflect
back to the parents and caretakers the measurements. A non-profit
corporation called LENA developed a vest that toddlers could wear which
would measure words spoken to them and words spoken with them without
recording the content of the conversations --- so that families would be
willing to have their toddlers wear these vests over multiple hours in
their homes. LENA also developed curricula --- slides, day by day
talking points and activities --- so that people could help parents
interpret and use the data on their children. The proposal was that
cities should buy LENA devices and also buy LENA's curriculum (books,
etc..) and use them with people who could visit families and discuss the
interpret the data. The idea was to give each family a 10 week course of
weekly visits and coaching of this kind --- including bringing
children's books to the home and discussions of how to read with your
child, topics of the week (like ``This week point out types of vehicles:
car, truck, bike.''), etc.. Thus, parental behavior would change ---
they would speak more to and with their toddlers. Thus, toddler language
developement would equalize between low and higher income families. And
thus, those children would arrive at kindergarten better able to take
advantage of the academic opportunities offered by the schools. This was
(and is) the hope and expectations. Hope based on values of equality and
social justice but also on large literatures on language development, on
behavior change and feedback, and on literatures that have developed on
precisely this topic (citations counts etc\ldots)

\textbf{Preliminary aparent success:} A team of academics and early
childhood learning expertes rolled out this program in Providence, RI in
20XX (the ``Providence Talks'' program). The (report){[}{]} used an
observational study design and regression analysis to report success:
that report showed families who recorded their home language
environments and who received visits from coaches using the LENA
curriculum improving their language home environment more than families
who had neither coaching nor word-recording.

In response, the Bloomberg Philanthropy decided to fund a replication
study involving RCTs in five other US cities: Detroit, Hartford,
Louisville, Birmingham, Virgina Beach. The Policy Lab at Brown
University would coordinate the studies. This paper is not the story of
that project: it stopped in the middle of COVID as it became more and
more difficult to send coaches into the homes of people with small
children (let alone implement the other features of the program ---
including play groups, and coaching of day care providers). But this
paper engages with some of the questions that arose that were not so
easy to answer using what we knew from the scholarly literature.

For example, we had to ask why would policy decisionmakers or the
scientific community want to replicate this study? Wouldn't it be enough
to read the reports written by the non-profit corporation who developed
and sell the technology? Or wouldn't it be enough to read the
observational study based in and around Providence, RI?

What did they hope to gain? What should anyone hope to gain? How much
replication? For what purpose?

We claim that they really want to improve their confidence in
explanations, and perhaps to fine tune them. And that they are aware
that ``causality operates in context''.

We also point out that a desire for replication here is a desire for
increased believability of findings. Certain findings do not seem to
help us increase (or decrease) the beliefs that we may previously hold.

There are at least two kinds of policy actors here (here focusing on
people with direct responsibility for early childhood public programs
for low income children, for example, not focusing on national or state
level policy activitists and lobbyists or pundits etc..):

So, why would someone in charge of early childhood programs in
Birmingham want to replicate the Providence study using an RCT instead
of an observational study in Birmingham?

Why would someone in city other than those chosen for the replication,
say, Urbana, find the results of 1 city observational study plus a 5
city loosely coordinatd RCT better for the purposes of informing their
decisions about Urbana, then just the single study?

While we can explain why someone might prefer an RCT over an
observational study (citations) easily: the families in the Providence
Study who were not offered the word-recording and coaching differed from
the families who did receive coaching and recording in many ways ---
ways that the researchers measured and attempted to adjust away using
regression models, and ways that the researchers did not measure and so
could not adjust away. We are left wondering both (1) whether the
families who participated in Providence were particularly likely to
benefit from the bundle of the intervention
(Coaching/Recording/Feedback) compared to those living just outside
Providence who did not get it, (2) whether language development in the
families who participated would have been faster and better over the
months of the study compared to the other non-participating families
because those focal, or interevention-families, cared more about toddler
language development in the first place --- thus they signed up for the
study. In an RCT we could have set aside such worries. We would have
been able to report on what we had learned with more confidence --- that
the treatment bundle caused differences in language development (or at
least, the treatment bundle and anything that came along with it).

So, we can see why those responsible for low-income early childhood
education in other city, including Providence, would prefer a
replication that is an RCT. We can also see how the RCT would raise new
questions for them (see Cartwright and hardie): if people could choose
to participate or not, would we see that the program was less effective?
(Recall that decision makers must evaluate this program in the context
of others --- each one costs money and time. Recording words counts
without storing recordings of words, for example, requires devices and
software and IT support. Coaching requires coaches and training and
materials.) That is, would familie who want to focus on their home
language environment sign up, but these families would be the ones who
would do other encirchment activities even in the absence of the
program? The RCT on its own does not tell us about its effect on the
kinds of families who would relish it versus avoid it. Nor does it help
decision makers figure out how to bring it into the homes of people who
would not ordinarily want to sign up for it: families which do not think
that language learning in their home for their toddlers requires this
kind of effort, or who are focusing on other aspects of parenting and
life in general.

Further, imagine that the debate in the community of those working to
improve early childhood language learning among low income toddlers is
between spending money on recording words and sending coaches to homes
versus just giving cash to families. The later idea arises from the idea
of guaranteed income, unconditional cash tranfers in development
economics, etc\ldots{} Would an RCT in Providence inform this debate?

But at least one RCT tells us something clear; something that can be a
base for other discussion.

Imagine we had an RCT from Providence. Why would folks in Birmingham
want to replicate it in Birmingham? Why would folks in Urbana be
grateful for a replication in Birmingham (or would they?)

The desire to replicate in Birmingham has to do with three factors: (1)
differences in treatment and (2) differences in the way context might
change the treatment effect and (3) how (and when) outcomes are
measured.

Imagine that services for low income families with toddlers in
Providence was provided from a social services agency that used social
workers and in Birmingham it would be provided via home-visiting nurses.
Would coaching and word-count data interpretation from a nurse who was
visiting to check the health of the child differ from the way a social
worker might provide those services? In fact, as we planned this
experiment we learned that each of the five cities has (1) pre-existing
processes and agencies to provide services to low income families, low
income toddlers and (2) that they all differ from one another. These
places each had a story about why their methods (say, bidding out
contracts to multiple neighborhood based small organizations, relying on
nurses who follow women from before birth to a couple of years after
birth, etc.) imply a very different approach to implementing the
word-counting and coaching, and that thus that their city would expect a
different result than any other city.

The decision makers in each city not only worried that their very
structures for implementing the new policy intervention differed, but
that the context of their cities differed: poverty in one city might be
concentrated among relatively recent Spanish speaking immigrants, while
it might be concentrated in historically marginalized neighborhoods
segregated by Black/White race in the US South. Of course the
word-counter technology adapted to Spanish word use. But, simple
comparisons of numbers of words, for example, might not be meaningful
given differences in language. Further, the families relationships are
socially and culturally constructed differently, jobs outside the home
for mothers differed in participation by women, and the relationships
between the city governments and public services and these groups
differed in terms of cooperation and suspicion.

Third, different cities measure language ability among their school
children at different ages --- not every place measures in kindergarten.
Among those who measure at the same age, we discovered that different
school districts and state-level education agencies mandate different
tests. The experts in childhood language in the room even disagreed on
which test measured concepts like ``kindergarten language ability'' best
(among native English speakers).

Yet, all agreed to the replication. Why?

One reason is that they were given money for their own individual
city-based and individual city-designed programs on the condition that
they do an RCT in a coordinated fashion.

A second reason was that each city feared that they could not serve
enough families in order to detect a treatment effect from noise. It is
costly to visit families in their homes. If a city only had 10 nurses,
those nurses could only visit maybe 100 families over the course of 10
weeks. Would 50 families in the treatment group and 50 in the control
group allow researchers to distinguish a treatmetn effect from zero, to
announce ``statistically significant'' for that given city? One idea was
that if the experiment could be analyzed as if it were done in 5 blocks
--- with say, a total of 250 families in treatment and 250 in control
--- then the statistical results would be stronger. That is, then, at
the level of the nation or for Bloomberg, the group of cities would be
able to contribute to an overall assessment of the benefits of this
treatment. Would 50 families in the treatment group and 50 in the
control group allow researchers to distinguish a treatmetn effect from
zero, to announce ``statistically significant'' for that given city? One
idea was that if the experiment could be analyzed as if it were done in
5 blocks --- with say, a total of 250 families in treatment and 250 in
control --- then the statistical results would be stronger. That is,
then, at the level of the nation or for Bloomberg, the group of cities
would be able to contribute to an overall assessment of the benefits of
this treatment.

Would 50 families in the treatment group and 50 in the control group
allow researchers to distinguish a treatmetn effect from zero, to
announce ``statistically significant'' for that given city? One idea was
that if the experiment could be analyzed as if it were done in 5 blocks
--- with say, a total of 250 families in treatment and 250 in control
--- then the statistical results would be stronger. That is, then, at
the level of the nation or for Bloomberg, the group of cities would be
able to contribute to an overall assessment of the benefits of this
treatment.

Would 50 families in the treatment group and 50 in the control group
allow researchers to distinguish a treatmetn effect from zero, to
announce ``statistically significant'' for that given city? One idea was
that if the experiment could be analyzed as if it were done in 5 blocks
--- with say, a total of 250 families in treatment and 250 in control
--- then the statistical results would be stronger. That is, then, at
the level of the nation or for Bloomberg, the group of cities would be
able to contribute to an overall assessment of the benefits of this
treatment.

We have a good sense for when an identical experiment would yield the
same result in a new place or time, and/or how the results of the
original experiment would differ from a new experiment or observational
study done with the same intervention and outcome in a new place and/or
time. The key in all cases is to have a sense of the mechanism linking
the intervention and outcome and the ability to measure variables that
drive that mechanism in both places. Given a theory linking an
intervention and an outcome and other variables that can be written as a
DAG, one can determine whether, say, the same intervention done in a new
context will yield the same results. Of course, we may be wrong.

Striking match --\textgreater{} Flame (Iteration 1. Imagining the
discovery of matches rather than their creation.)

Striking match --\textgreater{} Friction --\textgreater{} Flame
(Iteration 2)

Will it work outside our habitat on the Moon? Well, it depends on
measuring friction? Can you exert enough through the glove of your space
suit?

Turns out that it doesn't work despite multiple experiments with
different friction producing substances and forces. Why? The theory is
incomplete. In this case, a better theory would have included more about
combustion --- and gas environment. In that case, with a better DAG, we
could have said, since we can measure oxygen both at the source
experiment and the new place (the Moon surface outside the habitat), we
can say that the results of this experiment will not transport.

Stricking match --\textgreater{} Friction ---\textgreater{} Flame
\textless-- Oxygen (Iteration 3)

or

\begin{Shaded}
\begin{Highlighting}[]
\NormalTok{graph TD}
\NormalTok{    A[Striking Match] {-}{-}\textgreater{}|f| B(Friction)}
\NormalTok{    B {-}{-}\textgreater{} C\{fa:fa{-}fire Flame\}}
\NormalTok{    D[Spark] {-}{-}\textgreater{} C}
\NormalTok{    E[Oxygen] {-}{-}\textgreater{} C}
\end{Highlighting}
\end{Shaded}

{[}

\begin{tikzcd}
    & {\text{Oxygen}} \\
    {\text{Striking Match}} & {\text{Friction}} & {\text{Wood}} & {\text{Flame}} \\
    & {\text{Spark}}
    \arrow[from=2-1, to=2-2]
    \arrow[from=2-2, to=2-3]
    \arrow[from=1-2, to=2-3]
    \arrow[from=3-2, to=2-3]
    \arrow[from=2-3, to=2-4]
\end{tikzcd}

{]}

Notice that if this is a DAG that represents the shared knowledge about
matches and flame between people both on Earth and on the Moon, then
Earth-bound researchers could provide relevant information without
striking matches. For example, they could create a flame some other way
(perhaps finding one after a lightening strike, or igniting one with an
electric spark) and manipulating oxygen. If they couldn't get a flame
without oxygen then it follows that match striking would not be
effective on the moon \textbf{even if they never tested the striking the
flame} theory on the Earth let alone on the Moon.

\printbibliography[title=References]

\end{document}
