% Options for packages loaded elsewhere
\PassOptionsToPackage{unicode}{hyperref}
\PassOptionsToPackage{hyphens}{url}
\PassOptionsToPackage{dvipsnames,svgnames,x11names}{xcolor}
%
\documentclass[
  11pt,
]{article}
\usepackage{amsmath,amssymb}
\usepackage{lmodern}
\usepackage{iftex}
\ifPDFTeX
  \usepackage[T1]{fontenc}
  \usepackage[utf8]{inputenc}
  \usepackage{textcomp} % provide euro and other symbols
\else % if luatex or xetex
  \usepackage{unicode-math}
  \defaultfontfeatures{Scale=MatchLowercase}
  \defaultfontfeatures[\rmfamily]{Ligatures=TeX,Scale=1}
\fi
% Use upquote if available, for straight quotes in verbatim environments
\IfFileExists{upquote.sty}{\usepackage{upquote}}{}
\IfFileExists{microtype.sty}{% use microtype if available
  \usepackage[]{microtype}
  \UseMicrotypeSet[protrusion]{basicmath} % disable protrusion for tt fonts
}{}
\makeatletter
\@ifundefined{KOMAClassName}{% if non-KOMA class
  \IfFileExists{parskip.sty}{%
    \usepackage{parskip}
  }{% else
    \setlength{\parindent}{0pt}
    \setlength{\parskip}{6pt plus 2pt minus 1pt}}
}{% if KOMA class
  \KOMAoptions{parskip=half}}
\makeatother
\usepackage{xcolor}
\IfFileExists{xurl.sty}{\usepackage{xurl}}{} % add URL line breaks if available
\IfFileExists{bookmark.sty}{\usepackage{bookmark}}{\usepackage{hyperref}}
\hypersetup{
  pdfauthor={Jake Bowers},
  colorlinks=true,
  linkcolor={Maroon},
  filecolor={Maroon},
  citecolor={Blue},
  urlcolor={Blue},
  pdfcreator={LaTeX via pandoc}}
\urlstyle{same} % disable monospaced font for URLs
\usepackage[left=1.25in,right=1.25in,bottom=1in,top=1in]{geometry}
\setlength{\emergencystretch}{3em} % prevent overfull lines
\providecommand{\tightlist}{%
  \setlength{\itemsep}{0pt}\setlength{\parskip}{0pt}}
\setcounter{secnumdepth}{2}
\usepackage{fancyhdr,quiver,float}
\fancypagestyle{myfancy}{%
    \fancyhf{}
    \renewcommand{\headrulewidth}{0pt}
    \fancyfoot[R]{Version of --- \today --- \thepage}}
\pagestyle{myfancy}
\ifLuaTeX
  \usepackage{selnolig}  % disable illegal ligatures
\fi
\usepackage[style=authoryear-comp,]{biblatex}
\addbibresource{../../Research-Group-Bibliography/big.bib}

\title{What is the role of coordinated studies in evidence-based policy
making?}
\author{Jake Bowers\footnote{\href{mailto:jwbowers@illinois.edu}{\nolinkurl{jwbowers@illinois.edu}}
  Thanks to Jim Kuklinski (University of Illinois), Carrie Cihak (King
  County, WA), James Diossa (The Policy Lab @ Brown), and participants
  in The Policy Lab @ Brown University Providence Talks Replication
  project and meetings.}}
\date{September 26, 2021}

\begin{document}
\maketitle

\begin{abstract}

Coordinated randomized trials can help convince researchers that certain kinds
of causal explanations are more or less plausible: helping elaborate a shared
causal model linking interventions to outcomes. Policy makers may not share the
same causal model although coordinated studies combined with transparency
practices may convince policy makers to add nodes and paths to their model.

\end{abstract}

How can the advances in the cumulation of social science knowledge and
extrapolation of findings from one context to another help policy makers
make better decisions within their jurisdictions? As the evidence-based
public policy movement increases in strength and scope, many policy
experts are eager to ensure that tight budgets are efficiently and
effectively allocated --- so that money is not wasted, and most lives
are improved. How can the promises of this movement to be policy makers
be fulfilled by efforts in academia to coordinate experiments, to
increase transparency, and to better predict the results of studies done
in one place in another? This paper is motivated by the second question
and tries to answer the first.

To foreshadow the contributions here: I suggest that the challenges to
the use of, say, an average treatment effect and confidence interval
resulting from either a single study, a set of coordinated studies, or
an extrapolation using some weighting or prediction faced by a mayor or
a city council or some other policy making body are not currently well
addressed with current developments from academia, although those
developments lay the groundwork to overcome some of those challenges. It
seems to me that academia can help policy makers most, at this stage in
the development of the evidence-based policy movement, in the
development of theoretical frameworks explaining the mechanisms between
possible inputs and desired outputs. This effort may be helped by
coordination among policy-maker+academic teams, and will certainly be
enhanced by academic studies to help everyone learn about the proposed
causal connections that policy makers aim to use to wield the power of
government for the public welfare.

\hypertarget{the-problem-of-a-single-study-and-current-academic-and-policy-responses.}{%
\section{The problem of a single study and current academic and policy
responses.}\label{the-problem-of-a-single-study-and-current-academic-and-policy-responses.}}

A group of policy makers must make a policy and oversee its
implementation in a given place and time.\footnote{I'm imagining a city
  council or other executive group in this paper.} This policy is a
change to some process that the policy makers hope will improve some
outcomes in their jurisdiction. For example, they might hope to improve
both the mental and physical health, comfort, and neighborhood attitudes
of urban neighborhoods by planting trees along the sidewalk. Or they
might hope to improve the kindergarten readiness of low income children
by funding a program to coach families in how to talk and read with
their toddlers. Or they might make all public transit free in their city
in order to reduce inequality and improve the economy. In each case they
have choices about how to pursue these outcomes. And they also face
trade-offs across different areas.

Studies of the effects of coaching on kindergarten readiness or trees on
attitudes should, in principle, help policy makers. Or at least, many
researchers justify their research to themselves and others on the
grounds of policy relevance. And, to date, a fan of evidence-based
policy and a proponent of a new approach might say: ``This is a
promising approach that I heard about that worked in City B.'' and then
explaining, ``The finding was statistically significant. I think it was
an RCT.'' And others then asking whether and how the implementation of
that approach might need to be adapted to City A and whether the context
of City A might cause the effect of the new approach (say, planting
trees, coaching the families of toddlers, or free public transportation)
to have different effects and consequences than those new approaches had
in City B.

However, even as people with a priori reasons to be attracted to a given
new approach discuss implementation problems, they also know that
researchers can criticize the results of any single study on many
grounds --- if the results do not agree with the priors of some other
group, such opponents might dismiss the results if the analysis was not
done in a transparent way (for example if the analysis plan was not
pre-registered or the code and data were not made public). Or they might
argue that even a \(p < .01\) for a null hypothesis of no effects merely
means that a given result could be a rare event --- yes, surprising from
the perspective of the null hypothesis, but making an rare error doesn't
change the cost that error on the public. If the result arose in a study
in a context very different from the given city, critics might argue
that, say, a positive result in City B ought not be used as evidence on
which to base a costly policy change in City A.

Knowing about these criticisms, but still convinced that scientific
research should play a role in policy decision making, researchers have
developed a series of changes in practice. So, for example, a growing
number of evidence-based policy groups pre-register their analyses,
carefully document the complex flows of data files and code files that
merge, clean, and analyze. These groups also try to share as much of
their data as possible and engage in internal quality checks like blind
re-analysis of results and/or code-reviews (where as much as possible
done so as to be available to the public). Since many of the studies
done by these groups involve randomization, they tend to rely on
design-based statistical tests so as to protect their statistical
inferences from false positive errors arising from statistical technique
choices.

Although those efforts have increased the utility of the policy
evaluations those groups produce by reducing the plausibility of some
criticisms, they have still left the concerns about the surprisingly
large result from a single study arising by chance and the concerns that
a large, positive causal effect in one context might be small, or
negative, in another context.

\hypertarget{the-promise-of-coordinated-randomized-studies}{%
\section{The promise of coordinated randomized
studies}\label{the-promise-of-coordinated-randomized-studies}}

Coordinated randomized studies aim to address the two remaining
criticisms above and provide some other benefits to scientific knowledge
cumulation. I will describe them briefly here but note below that, to be
useful for a policy decision in City A, even responding to all of the
criticisms above is not enough.

If two studies investigate the \(Z \rightarrow Y\) relationship and
measure both \(Z\) and \(Y\) the same way, manipulate \(Z\) in the same
way, and occur in the same context (place and time), then we can respond
to the first set of concerns about confusing chance events with causal
effects.

If two studies occur in different contexts, but, as much as possible,
coordinate measurement and intervention, then we create information to
address the second set of concerns about heterogeneity of causal
effects. Of course, differences in detected causal effects between the
two studies could also arise from chance and not context. So, adding
more studies helps us distinguish between context-based differences in
effect and chance differences. (TODO: Elaborate on this a bit)

In these two kinds of coordinated experiments --- where effort goes into
coordinating measurement and intervention --- another benefit arises: an
overall causal effect of \(Z\) on \(Y\) can be estimated by combining
the studies, thereby increasing the statistical precision of the
estimation of this causal effect over that possible in any single study.
For example, one can treat the multiple experiments as a single
block-randomized experiment and one can estimate an average causal
effect for the units in all the studies.\footnote{If the causal effects
  might vary greatly across blocks, say, from strongly negative to
  positive, the average causal effect could be zero. This would be valid
  for that estimand. Of course, one could also assess the
  \(Z \rightarrow Y\) causal relationship using other tools which would
  report a strong relationship in this case.}

TODO: Either side step or talk about observational studies where
coordination has yet more benefits in helping us manage the biases
inherent in non-randomized studies. See Rosenbaum's 2021 Replication and
Evidence Factors in Observational Studies.

So, if we can address all of the criticisms that opponents of a policy
like planting trees, coaching families, or free public transit using
strong research transparency and integrity practices combined with
coordinated studies, should the policy makers in City A now have an
easier time making decisions than they did before these developments?

\hypertarget{challenges-in-the-use-of-coordinated-studies-by-policy-makers}{%
\section{Challenges in the use of coordinated studies by policy
makers}\label{challenges-in-the-use-of-coordinated-studies-by-policy-makers}}

Yes. The evidence in the hands of the policy makers in City A is
certainly more useful if it arrives from the kinds of processes just
described. Yet, the criticisms and responses just described left out the
political and human costs of error on the part of the policy makers.
They also ignored the concern that, even if an intervention appeared
powerful in multiple contexts, and even if it was predicted to be
powerful in City A using state of the art causal extrapolation
approaches, it might be too small to be worth doing given the other
policy options vying for attention and budget among the leaders of City
A.

Here is a list of some of the challenges facing those aiming to
influence the kinds of outcomes listed above as examples in their use of
the results of coordinated studies. Two conversations with policy makers
crystallized many of these points for me, which I had also observed as I
participated in the design of a coordinated experiment across 5 cities
on coaching and kindergarten readiness. So, I write this list as
questions and answers even even if this is not a record of any single
dialogue.

Q: What is the use of a coordinated study done in Cities B,C,D,E and F
for City A?

A: If one or more of those cities is similar to City A, then the results
\textbf{for that or those cities} would be used if the policy
intervention is realistic for City A. Results in not-similar cities or
cities where the intervention differed from the possible intervention
for City A would be ignored for fear that they would be misleading in
regards the possible effect in City A.

However, if an overall result, calculated using all of the cities,
convinced a philanthropic foundation to fund a not-harmful intervention
in City A, then, as long as the intervention did not face political
opposition, City A would be willing to try to create a locally relevant
implementation of the intervention.

Q: What about state-of-the art extrapolation from the not-similar cities
to City A using covariates measured in both places?

A: Conditional on the intervention being politically and practically
plausible and covariates measured the same way, with the same meaning in
both place, then if such tools confirmed prior beliefs, then they would
not be rejected. Errors are so costly that proponents of such tools have
a lot of work to do to convince policy makers of their utility, let
alone to overcome the prior concerns about the meaning and measurement
of covariates.

Q: What would it mean to use a result?

A: A result inspires a discussion about how to adapt the approach in the
other places to the local context. If those implementing the new
approach design a randomized trial to evaluate its success, they will
use measurements and interventions that are easy for them and which help
them interpret the causal effect estimated in their context. For
example, different school districts measure verbal ability using
different tests, and at different times in the life of a child: some use
Test A in Kindergarten, others Test B in 3rd grade. The school districts
have contracts with the makers of those tests and/or the tests are
chosen at the level of the U.S. State. To implement the same test at the
same time for thousands of children across districts dramatically
increases the cost of a study and provides unclear information within
localities which have processes already existing.

Q: What might prevent use of a result?

A: City A would need to be convinced that the kinds of people that the
intervention aims to help were not harmed and in fact benefited from the
intervention even if people benefited on average. For example, some
jurisdictions make racial equity a criteria for policy interventions. A
finding that, on average, people benefited but people of color did not
benefit, would prevent the use of the result. Above and beyond the
subgroup specific effects, City A might also need to claim a kind of
``no harm'' expectation when building a coalition to support the new
policy.

The preceding fake conversation reveals that concerns about causal
models/mechanisms for interventions, concerns about distributional
consequences, and concerns about local implementation possibilities
could all prevent the use of a given piece of evidence by policy makers.

Policy makers have implicit (or explicit) causal models relating
interventions to outcomes. For example,
Figure\textasciitilde{}\ref{fig:theory2} shows a stylized causal model
that a policy maker might have of kindergarten readiness in their own
city (thus no \(\boxed{S}\) below). The coaching intervention might
improve kindergarten readiness, but so might addressing Family SES,
Neighborhood Poverty, Teacher Quality, or Classroom sizes.

\begin{figure}[H]
\centering
\begin{tikzcd}[every arrow/.append style=-latex]
{\text{Family SES}}  \arrow[from=1-1, to=1-2] \arrow[from=1-1, to=2-1] & {\text{Kindergarten Readiness Verbal Skills}} \\
{\text{Home Language Environment Quality}}   \arrow[from=2-1, to=1-2] \\
{\text{Neighborhood Poverty}}  \arrow[from=3-1, to=1-2] \\
{\text{Pre-School Classroom Size}}       \arrow[from=4-1, to=1-2] \\
{\text{Pre-School Teacher Quality}} \arrow[from=5-1, to=1-2] \\
   \end{tikzcd}
\caption{A theory of educational outcomes at the level of concepts and
relationships.}\label{fig:theory2}
\end{figure}

Before hearing about the coordinated coaching study, the policy makers
in City A might not have had the ``Home Language Environment Quality''
as a part of their causal model. But upon hearing about the study, they
might be willing to elaborate their causal model to add new elements to
it. For example, they might be willing to believe that the intervention
caused the outcome in the other cities given the benefits of
transparency and coordination mentioned earlier.

Once the new element is added to the causal model, questions about
implementation and distribution arise. Should they implement the
coaching approach that appears promising from a study that ensured that
all of the coaching interventions were done the same way across multiple
cities? City A might do so only if they could imagine a plausible mode
of implementing that coaching intervention (where ``plausible'' means
not creating political conflict, and not costing more than intervening
in any of the other parts of the causal model). For example, if home
visiting nurses were the coaches in the cities in the coordinated study,
and if City A had an active home visiting nurse program for low income
families, the implementation of the program in City A might appear
easier and plausible. However, if City A had decided long ago to not use
home visiting nurses, and provided similar services in a different way,
then City A might find the results of the coordinate study less
relevant.

Imagine the City A did have a home visiting program that aimed to
support low income families --- in City A, imagine that these people
were mostly agricultural workers. But the coordinated study did not
report results for such people. City A would then wonder about whether
the positive result from the coordinated study would be useful for its
population.

Further, imagine that results from the coordinated experiment were
broken down within each city by income and type of employment. The
decision makers in City A would ask whether coaching from nurses in City
A would have the same kinds of effects as they would in the other
Cities. Would living in apartment buildings change the effect? Or would
home language matter? Would family structure matter? Causal mechanisms
operate in a context.

So, what is the benefit of the coordinated experiment for the policy
makers in City A? It appears to be mostly to encourage them to add an
element to their causal model. It could also encourage them to build
their own intervention inspired by the results of the coordinated
experiment, and perhaps even to have their own evaluation of their own
new intervention. However, there are many barriers between a report from
a coordinated experiment and the implementation of policy that appeared
successful in other cities in City A.

\hypertarget{discussion-and-next-steps-in-evidence-based-policy-making}{%
\section{Discussion and next steps in evidence-based policy
making}\label{discussion-and-next-steps-in-evidence-based-policy-making}}

Coordinated studies and transparency practices help researchers add and
remove paths from their causal model. In turn, this kind of scientific
consensus about the nodes and paths of a model can help policy makers
elaborate their own causal models (often also known as ``theories of
change''). This appears to be one of the most policy-relevant benefits
of these studies from my current small set of interviews and
observations.

What else do policy makers need? Focused studies on key elements of the
causal model could help. They need not be coordinated, but should
explicitly target a causal pathway of interest to a policy-maker.
Studies of variation in causal effects across different contexts could
also help policy makers judge the relevance of any given finding for
their jurisdiction.

What do we hope from the design of studies, coordinated or not? One hope
is that such studies would convince policy makers to update their prior
beliefs. The strongest such study would convince people who held beliefs
that ran opposite to the findings of the study.

\printbibliography

\end{document}
