% Options for packages loaded elsewhere
\PassOptionsToPackage{unicode}{hyperref}
\PassOptionsToPackage{hyphens}{url}
\PassOptionsToPackage{dvipsnames,svgnames,x11names}{xcolor}
%
\documentclass[
]{article}
\usepackage{amsmath,amssymb}
\usepackage{lmodern}
\usepackage{iftex}
\ifPDFTeX
  \usepackage[T1]{fontenc}
  \usepackage[utf8]{inputenc}
  \usepackage{textcomp} % provide euro and other symbols
\else % if luatex or xetex
  \usepackage{unicode-math}
  \defaultfontfeatures{Scale=MatchLowercase}
  \defaultfontfeatures[\rmfamily]{Ligatures=TeX,Scale=1}
  \setmainfont[]{Merriweather-Light}
\fi
% Use upquote if available, for straight quotes in verbatim environments
\IfFileExists{upquote.sty}{\usepackage{upquote}}{}
\IfFileExists{microtype.sty}{% use microtype if available
  \usepackage[]{microtype}
  \UseMicrotypeSet[protrusion]{basicmath} % disable protrusion for tt fonts
}{}
\makeatletter
\@ifundefined{KOMAClassName}{% if non-KOMA class
  \IfFileExists{parskip.sty}{%
    \usepackage{parskip}
  }{% else
    \setlength{\parindent}{0pt}
    \setlength{\parskip}{6pt plus 2pt minus 1pt}}
}{% if KOMA class
  \KOMAoptions{parskip=half}}
\makeatother
\usepackage{xcolor}
\IfFileExists{xurl.sty}{\usepackage{xurl}}{} % add URL line breaks if available
\IfFileExists{bookmark.sty}{\usepackage{bookmark}}{\usepackage{hyperref}}
\hypersetup{
  pdfauthor={Jake Bowers; James Kuklinski; Carrie Cihak},
  colorlinks=true,
  linkcolor={Maroon},
  filecolor={Maroon},
  citecolor={Blue},
  urlcolor={Blue},
  pdfcreator={LaTeX via pandoc}}
\urlstyle{same} % disable monospaced font for URLs
\usepackage[left=1.25in,right=1.25in,bottom=1in,top=1in]{geometry}
\setlength{\emergencystretch}{3em} % prevent overfull lines
\providecommand{\tightlist}{%
  \setlength{\itemsep}{0pt}\setlength{\parskip}{0pt}}
\setcounter{secnumdepth}{2}
\usepackage{fancyhdr,quiver}
\fancypagestyle{myfancy}{%
  \fancyfoot[C]{}
  \fancyfoot[R]{Version of --- \today --- \thepage}}
\pagestyle{myfancy}
\ifLuaTeX
  \usepackage{selnolig}  % disable illegal ligatures
\fi
\usepackage[]{biblatex}
\addbibresource{../../Research-Group-Bibliography/big.bib}

\title{Decisions}
\author{Jake Bowers\footnote{\href{mailto:jwbowers@illinois.edu}{\nolinkurl{jwbowers@illinois.edu}}} \and James
Kuklinski \and Carrie Cihak}
\date{}

\begin{document}
\maketitle

\hypertarget{overviewintroduction}{%
\section{Overview/Introduction}\label{overviewintroduction}}

A mayor has decided to work to improve the educational outcomes of low
income children in her city. Inspired by the success of the Providence
Talks pilot study, and the literature on the ``word gap'', she decides
to try to improve the verbal abilities of children from low income
families before they enter kindergarten.

\begin{tikzcd}[every arrow/.append style=-latex]
\text{Home Language Environment Quality} \arrow[rd]                                              &             \\
\text{Family SES} \arrow[r] \arrow[u] & \text{pre-Kindergarten Verbal Ability}          \\
\text{Neighborhood Poverty} \arrow[ru] \arrow[uu, bend left=69] \arrow[u] \arrow[r] & \text{Pre-School Quality} \arrow[u]
\end{tikzcd}

\hypertarget{a-decision-setting}{%
\section{A decision setting:}\label{a-decision-setting}}

A mayor has decided to work to improve the educational outcomes of low
income children in her city. She begins by learning about what social
scientists currently think cause such outcomes and comes up with a graph
to describe her results.

\begin{figure}[!h]
\centering
\begin{tikzcd}[every arrow/.append style=-latex]
 {\text{Family SES}}  & {\text{Educational Outcomes}} \\
 {\text{Classroom Size}} \\
 {\text{Teacher Quality}} \\
 {\text{Neighborhood Poverty}} \\
    \arrow[from=1-1, to=1-2]
    \arrow[from=2-1, to=1-2]
    \arrow[from=3-1, to=1-2]
    \arrow[from=4-1, to=1-2]
\end{tikzcd}
\caption{A theory of educational outcomes at the level of concepts and
relationships.}\label{fig:theory1}
\end{figure}

While puzzling over what she could do with her limited budget (or what
she could convince the city council to support given the politics of
that body), she hears about the Providence Talks Program (the Mayor of
Providence, RI received an award for this program and the Bloomberg
Foundation was funding a 5 city replication of it, using RCTs instead of
the original observational study design). This was useful to learn
because it also highlighted that the first causal graph was much too
general --- Whose educational outcomes should matter? What age?
Obviously certain kinds of policy interventions focus on different parts
of the graph and aim at certain causal pathways and outcomes. The
Providence Talks program focused on verbal abilities among toddlers with
the aim that they would enter kindergarten well-prepared for school.
Early childhood education outcomes seemed like something that the mayor
would like to focus on this term --- maybe educational outcomes for
older groups could be the focus next term if she is re-elected or a
focus for some other person in her administration.

\begin{figure}[!h]
\centering
\begin{tikzcd}[every arrow/.append style=-latex]
{\text{Family SES}}  \arrow[from=1-1, to=1-2] \arrow[from=1-1, to=2-1] & {\text{Kindergarten Readiness Verbal Skills}} \\
{\text{Home Language Environment Quality}}   \arrow[from=2-1, to=1-2] \\
{\text{Neighborhood Poverty}}  \arrow[from=3-1, to=1-2] \\
{\text{Pre-School Classroom Size}}       \arrow[from=4-1, to=1-2] \\
{\text{Pre-School Teacher Quality}} \arrow[from=5-1, to=1-2] \\
   \end{tikzcd}
\caption{A theory of educational outcomes at the level of concepts and
relationships.}\label{fig:theory2}
\end{figure}

She now has some decisions to make:

\begin{itemize}
\tightlist
\item
  She could commission research in her own city (let's imagine that all
  research in this paper is an RCT just to make it a bit easier).
\item
  She could contribute to fund research in another city (say, this is
  cheaper and less disruptive of her own city). (the other city could be
  (a) ``like'' her city or (b) not very similar to her own city. Where
  similarity has to do with context --- and the savvy mayor is really
  thinking about context as different variables that could change or
  limit any of the arrows drawn above (i.e. imagine a ``context'' node
  in between each of the main causal drivers and the outcomes)
\item
  She could contribute to one of the coordinated experiments occuring
  around the world (where, say, similar interventions are done in
  multiple different cities --- none of them very similar to her own
  city)
\item
  She could try to get her city to participate in a coordinated
  experiment.
\end{itemize}

Now, there is a preliminary problem which is that ``educational
outcomes'' are too vague to be easily measured. And the literature
review in fact included studies of math scores among high school seniors
and studies of kindergarten readiness. She decides to focus on
kindergarten readiness and specifically on language abilities at
kindergarten entry because she read some essays that claim that language
abilities upon entering kindergarten can seriously disadvantage children
with effects that are very long term --- if those kids had not been
behind in reading and speaking and writing, they would have not been
behind in math in their senior year of high school, or been behind in
high school graduate rates.

\printbibliography

\end{document}
